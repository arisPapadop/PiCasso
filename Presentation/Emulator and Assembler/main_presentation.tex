%%%%%%%%%%%%%%%%%%%%%%%%%%%%%%%%%%%%%%%%%
% Beamer Presentation
% LaTeX Template

%----------------------------------------------------------------------------------------
%	PACKAGES AND THEMES
%----------------------------------------------------------------------------------------
\documentclass{beamer}

\mode<presentation> {

\usetheme{Madrid}

\setbeamertemplate{navigation symbols}{}
}

\usepackage{graphicx} % Allows including images
\usepackage{booktabs} % Allows the use of \toprule, \midrule and \bottomrule in tables

%----------------------------------------------------------------------------------------
%	TITLE PAGE
%----------------------------------------------------------------------------------------

\title[Main Project]{ARM11: Emulator \& Assembler} % The short title appears at the bottom of every slide, the full title is only on the title page

\author[Group MSB]{
  \textbf{Group MSB (aka 32)}\\
  \and
  	Ramon Fern\'{a}ndez i Mir\\
  \and
  	Robert Holland\\
   \and
	Jennifer Lea \\
  \and
  	Aris Papadopoulos\\
}
\institute[ICL] {

Imperial College London \\ 

\medskip
}
\date{June 17, 2016}

\begin{document}

\begin{frame}
\titlepage
\end{frame}

\begin{frame}
\frametitle{Overview} 

\tableofcontents 

\end{frame}

%----------------------------------------------------------------------------------------
%	PRESENTATION SLIDES
%----------------------------------------------------------------------------------------

%------------------------------------------------
\section{Emulator}
%------------------------------------------------

%------------------------------------------------
\subsection{The Pipeline} 
%------------------------------------------------
\begin{frame}
\frametitle{Emulator: The Pipeline}

\begin{large}
Our Aims:
\end{large}

\begin{itemize}

\item Code that actually \textbf{resembles} the function of the ARM11 Control Unit \\

\item \textbf{Scalable} code, for the addition of new instruction types \\

\item \textbf{Readable} and understandable code \\

\item \textbf{Efficient} code, in terms of group work

\end{itemize}

\end{frame}

%------------------------------------------------

\begin{frame}[fragile]
\frametitle{Emulator: The Pipeline}

\begin{columns}

\begin{column}{0.05\textwidth}
\end{column}

\begin{column}{0.80\textwidth}
\begin{block}{The main loop}
\begin{verbatim}
do {
  execute();
  decode();
  fetch();
} while (pipeline.decoded.type != HALT);
\end{verbatim}
\end{block}
\end{column}

\begin{column}{0.05\textwidth}
\end{column}

\end{columns}

\end{frame}
%------------------------------------------------
\subsection{Instructions}
%------------------------------------------------
\begin{frame}[fragile]
\frametitle{Emulator: Instructions}

\begin{columns}

\begin{column}{0.05\textwidth}
\end{column}

\begin{column}{0.90\textwidth}
\begin{block}{Instruction Structure}
\begin{verbatim}
typedef struct Instruction {
  uint8_t          cond;
  Instruction_Type type;

  Data_Processing      data_processing;
  Multiply             multiply;
  Single_Data_Transfer single_data_transfer;
  Branch               branch;
  Left_Shift           left_shift;
} Instruction;
\end{verbatim}
\end{block}
\end{column}

\begin{column}{0.05\textwidth}
\end{column}

\end{columns}

\end{frame}

%------------------------------------------------
\subsection{Test Driven Design}
%------------------------------------------------
\begin{frame}
\frametitle{Emulator: Test Driven Design}

\begin{itemize}
\item Printing
\item Word building
\item CPSR
\item GPIO
\end{itemize}
%------------------------------------------------
\end{frame}

%------------------------------------------------
\section{Assembler}
%------------------------------------------------

%------------------------------------------------
\subsection{Two-Pass Assembly}
%------------------------------------------------

\begin{frame}[fragile]
\frametitle{Assembler: Two-Pass Assembly}

\begin{columns}

\begin{column}{0.07\textwidth}
\end{column}

\begin{column}{0.80\textwidth}
\begin{block}{First pass pseudocode}
\begin{verbatim}
while (more lines exist) {
  line = get_line();
  if (is_label(line)) {
    put_address(line, line_address);
  } else if (not empty line) {
    line_address += WORD_SIZE;
  }
}
end_of_file = line_address;
\end{verbatim}
\end{block}

\begin{itemize}
\item Put all labels in a hash-table, to be used for assembling branch instructions.
\item Store the number of lines with instructions, to know where the file ends.
\end{itemize}
\end{column}

\begin{column}{0.07\textwidth}
\end{column}

\end{columns}
\end{frame}
%------------------------------------------------

\begin{frame}[fragile]
\frametitle{Assembler: Two-Pass Assembly}

\begin{columns}

\begin{column}{0.07\textwidth}
\end{column}

\begin{column}{0.80\textwidth}
\begin{block}{Second pass pseudocode}
\begin{verbatim}
while (more lines exist) {
  line = get_line(); 
  if (!is_label(line) && !is_empty) {
    Token instr = tokenize(line);
    
    Instruction encoded 
         = get(mnemonic)->function(arguments);
   
    uint32_t assembled = assemble(encoded);
    write_instruction();
   
    current_address += WORD_SIZE;
  }
}\end{verbatim}
\end{block}
\end{column}

\begin{column}{0.07\textwidth}
\end{column}

\end{columns}

\end{frame}
%------------------------------------------------
\subsection{Hash-Table}
\begin{frame}[fragile]
\frametitle{Assembler: Hash-Table}

\begin{large}
Strengths:
\end{large}

\begin{itemize}
\item No giant switch statement
\item Very scalable
\item Becomes even quicker as the instruction set grows
\item Fun to code
\end{itemize}

\begin{large}
Limitations:
\end{large}

\begin{itemize}
\item Can be as efficient as a linked list
\item Need to hardcode inputs
\item More complicated code
\end{itemize}

\end{frame}




%------------------------------------------------
\begin{frame}[fragile]
\frametitle{Assembler: Hash-Table}

\begin{block}{put\_function(char *mnemonic, Instr\_Ptr function)}
\begin{verbatim}
if ((curr = get(mnemonic)) == NULL) {
  curr = (Node *) calloc(1, sizeof(*curr));
  if (curr == NULL || 
     (curr->mnemonic = strdup(mnemonic)) == NULL) {
    return NULL;
  }
  hash_code = hash(mnemonic);
  curr->next = functiontable[hash_code];
  functiontable[hash_code] = curr;
} else {
  free((void *)(&(curr->function)));
}
if ((curr->function = function) == NULL) return NULL;
return curr;
\end{verbatim}
\end{block}

\end{frame}

%------------------------------------------------
\subsection{Test Driven Design}
%------------------------------------------------

\begin{frame}
\frametitle{Assembler: Test Driven Design}

\begin{itemize}
\item Memory handling
\item Shifting registers
\end{itemize}

\end{frame}
%------------------------------------------------

\end{document} 